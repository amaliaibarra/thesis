\begin{opinion}
Como uno de los tutores de la tesis titulada "Módulo fabric-chaincode-api para implementar contratos inteligentes empleando $C\sharp$" elaborada por la diplomante Amalia Nilda Ibarra Rodríguez; con el fin de obtener el título de Licenciado en Ciencias de la Computación, hago constar primeramente que el tema de investigación seleccionado es pertinente, oportuno y con resultados aplicables al problema del uso del lenguaje de programación $C\sharp$ en el desarrollo de solución con Hyperledger Fabric, teniendo como resultado un módulo extensible que permite de manera efectiva la implementación de contratos inteligentes usando el lenguaje mencionado. 

Se expresa que el autor ha mostrado disciplina y dedicación en la realización del ejercicio, tanto en la redacción del trabajo de diploma, como en la organización y la implementación de la solución, lo cual se ve reflejado en la revisión del producto entregado por el diplomante. Para ello, comenzó con la asimilación y estudio de las tecnologías indicadas para desarrollar el sistema, mostrando capacidad para manejar el entendimiento de componentes de bajo nivel de la tecnología utilizada. De igual modo mostró buenas capacidades de investigación.

En consecuencia, se define que la tesis cumple rigor metodológico, científico y está en función de los requisitos definidos, partiendo además del estudio de fuentes y publicaciones recientes relacionadas al tema de investigación.

Por tanto, hago constar que la tesis reúne los estándares metodológicos exigidos por la Facultad de Matemática y Computación de la Universidad de la Habana, para ser presentada y sometida a evaluación en su ejercicio de defensa.

Felicito a la diplomante por haber respondido con responsabilidad al desafío del estudio y haber finalizado exitosamente su trabajo de diploma.

--
Ing. Daniel Frias Mena

\end{opinion}