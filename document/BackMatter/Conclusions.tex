\begin{conclusions}
En el presente trabajo se realizó un resumen de los conceptos principales de Hyperledger Fabric. Se hizo énfasis en las características de diseño que hacen posible la programación de contratos inteligentes en lenguajes de propósito general. Cumpliendo con uno de los objetivos específicos de este trabajo, se llevó a cabo un estudio de las APIs existentes para Hyperledger Fabric que permiten escribir contratos inteligentes en otros lenguajes de programación. Se expusieron detalles del diseño e implementación de la biblioteca \texttt{fabric-chaincode-api} propuesta. Se muestran los resultados obtenidos mediante ejemplos de contratos inteligentes programados en $ C \sharp $. Por último se recomiendan tareas para mejorar la biblioteca propuesta en investigaciones futuras.

Los elementos anteriores contribuyeron al cumplimiento del objetivo general de este trabajo, el cual consiste en poder integrar la lógica de los contratos inteligentes programados en $ C \sharp $ dentro de la arquitectura de una \textit{blockchain} Hyperledger Fabric.
\end{conclusions}
