\chapter*{Introducción}\label{chapter:introduction}
\addcontentsline{toc}{chapter}{Introducción}

Hyperledger Fabric es una plataforma de tecnología de libro mayor distribuido (DLT) de código abierto, diseñada para su uso en contextos empresariales. Su arquitectura altamente modular y configurable permite la innovación, la versatilidad y la optimización para una amplia gama de casos de uso de la industria, incluidos la banca, las finanzas, los seguros, la atención médica, los recursos humanos, la cadena de suministro e incluso la entrega de música digital.

Entre las capacidades que distinguen a Hyperledger Fabric de otras blockchains destaca ser la primera plataforma DLT que admite contratos inteligentes creados en lenguajes de programación de propósito general como Java, Go y Node.js, en lugar de lenguajes específicos de dominio restringidos (DSL). Esto significa que la mayoría de las empresas ya tienen el conjunto de habilidades necesario para desarrollar contratos inteligentes y no se necesita capacitación adicional para aprender un nuevo idioma o DSL por tanto, es una gran ventaja. Lo novedoso es que la implementación de contratos inteligentes no está limitada a los lenguajes anteriores, es posible expandir la lista a cualquier lenguaje de programación. 

A partir de esta posibilidad el Instituto de Criptografía de la Universidad de la Habana con el propósito de facilitar el entrenamiento docente en la concepción de contratos traza como objetivo general integrar la lógica de los contratos inteligentes programados en $C\sharp$ dentro de la arquitectura de una blockchain Hyperledger Fabric.

Por qué $C \sharp$? Bueno, además de contar con una amplia comunidad de desarrolladores alrededor del mundo $C \sharp$ es el lenguaje con el que los estudiantes de la carrera de Ciencias de la Computación aprenden a programar. Poder crear contratos inteligentes en un lenguaje conocido es una ventaja que incentiva el acercamiento al mundo del Blockchain en los primeros años de la carrera.

El problema a resolver consiste entonces en poder programar contratos inteligentes en $C\sharp$ que interactúen con una Blockchain en Hyperledger Fabric.

Se propone construir una API fabric-contract-api-csharp,  que proporcione la interfaz del contrato, un punto de entrada de alto nivel para escribir la lógica empresarial.

Para el cumplimiento de este objetivo general se llevarán a cabo las siguientes tareas:

\begin{enumerate}
\item Estudio de las APIs existentes para los lenguajes actualmente soportados.
\item Diseño e implementación del Software Development Kit (SDK).
\item Desarrollo de un repositorio de contratos reutilizables.
\end{enumerate}

\subsection*{Presentación de los siguientes capítulos}	
